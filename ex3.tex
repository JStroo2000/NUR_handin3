\subsection*{c)}
In this section the results of subquestion c will be discussed. The script used for this subquestion is 
\lstinputlisting{NUR_handin3c.py}
As before, this script also supplies answers for question d.
The plots created by this script are:
\begin{figure}[!h]
    \centering
    \includegraphics{./plots/poissonm11.pdf}
    \caption{The poisson log-likelihood fit to the data of mass bin m11.}
    \label{fig:poissonm11}
\end{figure}

\begin{figure}[!h]
    \centering
    \includegraphics{./plots/poissonm12.pdf}
    \caption{The poisson log-likelihood fit to the data of mass bin m12.}
    \label{fig:poissonm12}
\end{figure}

\begin{figure}[!h]
    \centering
    \includegraphics{./plots/poissonm13.pdf}
    \caption{The poisson log-likelihood fit to the data of mass bin m13.}
    \label{fig:poissonm13}
\end{figure}

\begin{figure}
    \centering
    \includegraphics{./plots/poissonm14.pdf}
    \caption{The poisson log-likelihood fit to the data of mass bin m14.}
    \label{fig:poissonm14}
\end{figure}

\begin{figure}
    \centering
    \includegraphics{./plots/poissonm15.pdf}
    \caption{The poisson log-likelihood fit to the data of mass bin m15}
    \label{fig:poissonm15}
\end{figure}

The steps taken are as above: for each data file 50 logarithmic bins were created, over the same range as before. The downhill simplex method was then used to minimize the fit function, in this case the poissonian log-likelihood function
\begin{equation}
    -ln \mathcal{L}(\mathbf{p}) = -\sum_{i=0}^{N-1} \{y_iln[\mu(x|\mathbf{p})] - \mu(x|\mathbf{p}) - ln(y_i!)\}
\end{equation}

As above, the fits overestimate the data by multiple orders of magnitude. This at least confirms the issue lies with the minimisation method, as that is the only part of these scripts shared between only the chi squared minimization and the poissonian minimization. Regardless, the results aren't very good.
